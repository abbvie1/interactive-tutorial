\documentclass[11pt]{article}
\usepackage[utf8]{inputenc}
\usepackage[T1]{fontenc}
\usepackage{fullpage}
\usepackage{graphicx}
\usepackage{grffile}
\usepackage{longtable}
\usepackage{wrapfig}
\usepackage{rotating}
\usepackage[normalem]{ulem}
\usepackage{amsmath}
\usepackage{textcomp}
\usepackage{amssymb}
\usepackage{capt-of}
\usepackage{hyperref}
\author{Toby Dylan Hocking}
\date{\today}

\begin{document}

\section*{``Understanding and creating interactive graphics,'' a tutorial for useR 2016}
%\tableofcontents

%% \section*{{\bfseries\sffamily TODO} }
%% \label{sec:orgheadline1}

%% send this to useR-2016@R-project.org before January 10, 2016.

\label{sec:orgheadline3}

\begin{center}
\begin{tabular}{llll}
Instructor & Institution & Address & Email\\
\hline
Toby Dylan Hocking & McGill Univ.  & Montreal, Canada & toby.hocking@mail.mcgill.ca\\
Claus Thorn Ekstrøm & Univ. Copenhagen & Copenhagen, Denmark & ekstrom@sund.ku.dk\\
\end{tabular}
\end{center}

\section*{Background}
\label{sec:orgheadline5}

An interactive graphic invites the viewer to become an active partner
in the analysis and allows for immediate feedback on how the data and
results may change when inputs are modified. Interactive graphics can
be extremely useful for exploratory data analysis, for teaching, and
for reporting.

Because there are so many different kinds of interactive graphics,
there has recently been an explosion in R packages that can produce them
(e.g. animint, shiny, rCharts, rMaps, ggvis, htmlwidgets). A beginner
with little knowledge of interactive graphics can thus be easily
confused by (1) understanding what kinds of graphics are useful for
what kinds of data, and (2) finding an R package that can produce the
desired type of graphic. This tutorial solves these two problems by
(1) introducing a vocabulary of keywords for understanding the
different kinds of graphics, and (2) explaining what R packages can be
used for each kind of graphic.

\section*{Detailed Outline}
\label{sec:orgheadline10}

\subsection*{A vocabulary for understanding interactive graphics, 30 minutes}
\label{sec:orgheadline7}

We will use the following vocabulary to classify each interactive
graphic in terms of complexity, possible actions, and effects.

\begin{description}
\item[Complexity] multi-panel, multi-layer, multi-plot.
\item[Actions] animation, direct manipulation (clicking, hovering), indirect manipulation
  (menus, buttons).
\item[Effects] zoom, highlight, show/hide (data, labels, tooltips),
  hyperlink.
\end{description}

\subsection*{High-level interactive plotting packages, 30 minutes}
\begin{itemize}
\item Simple approaches like rotating plots (\textbf{rgl} package) and
  keyboard interaction (wallyplot from \textbf{MESS} package).
\item Interactive bar plots (\textbf{rCharts}, several different JavaScript
  interfaces, interfacing with JavaScript libraries to change axes
  and legends).
\item Interactive scatter plots showing happiness and tax rate
  (\textbf{rCharts}, \textbf{clickme}, several different JavaScript
  interfaces, add dropdown effects and improve tooltips).
\item interactive maps and choropleths (\textbf{rMaps}).
\item Discussion of frustrations that new users unfamiliar with
  JavaScript may encounter when interfacing with JavaScript libraries.
\end{itemize}

\subsection*{Interactive graphics with shiny and plotly, 30 minutes}
\begin{itemize}
\item Teaching least squares and power calculations (\textbf{shiny}).
\item Reproducing some of the previous graphics on happiness and tax
  rate (\textbf{plotly}+\textbf{ggplot2}, adding tooltips/hover
  effects, and dropdown).
\item Graphics on prediction accuracy for Danish population predictions
  (\textbf{plotly}, adding sliders).
\end{itemize}

\subsection*{Multi-layer graphics, ggplot2 package, 15 minutes}
\begin{itemize}
\item Map of tornadoes (points and segments) in the United States (polygons).
\item Change-point detection: data points, segment means,
  change-points.
\end{itemize}

\subsection*{Multi-panel graphics, facets in ggplot2, 15 minutes}
Facets are useful in two different situations:
\begin{description}
\item[{Same plot for different data subsets}] a linear model fit to each
  of several data subsets.
\item[{Different plots with aligned axes}] World Bank data viz with one
  time series panel, and one scatterplot panel.
\end{description}

\subsection{Animated graphics, animation package, 15 minutes}
\begin{itemize}
\item Gradient descent (time=iterations).
\item Two-panel World Bank data viz (time=years).
\end{itemize}

\subsection{Interactive + animated + multi-panel + multi-layer, 45 minutes}
few packages are able to produce complex graphics which can be
described by several vocabulary words.
\begin{description}
\item[{shiny + ggplot2}] World Bank data viz, interacting with widgets
changes selected year, countries, regions.
\item[{shiny + ggvis}] same kind of graphic with World Bank data.
\item[{animint}] World Bank data viz, direct manipulation changes
selected year, countries, regions.
\end{description}

\subsection*{Background Knowledge}
\label{sec:orgheadline12}

Since we plan to present state-of-the-art interactive graphics, people
should know how to use R data structures (lists, data.frames) and the
ggplot2 package. 

Even though many examples will be interactive web graphics, we will
assume only knowledge of R, not HTML/JavaScript.

There are two classes of potential attendees:
\begin{itemize}
\item UseRs who are not very familiar with interactive graphics should
benefit the most, since we will give a high-level overview of many
different packages.
\item DevelopeRs of interactive graphics packages are encouraged to
  come, to discuss the current state-of-the-art and future directions.
\end{itemize}

\end{document}
